%!TEX root = ../thesis.tex
%*******************************************************************************
%*********************************** Fourth Chapter ****************************
%*******************************************************************************

\chapter{Discussion} %Title of the First Chapter

\ifpdf
    \graphicspath{{Chapter4/Figs/Raster/}{Chapter4/Figs/PDF/}{Chapter4/Figs/}}
\else
    \graphicspath{{Chapter4/Figs/Vector/}{Chapter4/Figs/}}
\fi


%********************************** %First Section  **************************************
\section{section} 

Think about why \textit{C. flexa} has the behavior that it does. Given how other choanoflagellates like \textit{S. rosetta} get their morphology by division \citep{fairclough2010,larson2020}, it could be that \textit{C. flexa} forms by division and has evolved its contractile ring to take advantage of that. This would make sense from the perspective that choanoflagellate cells aligned and lined up next to each other drive the strongest flows (though still not stronger than they could individually) \citet{kirkegaard2016}. This paper also found that being farther from a wall increases flux. This is interesting considering that the feeding state of \textit{C. flexa} was observed to be so ineffective at swimming that it sank and remained in place. However this was on a slide and in the absense of external flows \citet{brunet2019}.

When thinking about \textit{C. flexa} in the context of multicellularity, we should not overlook the simplicity by which it achieves large-scale geometric changes. While we can develop increasingly complex models by introducing collar filament bending, tension and stress at the collar filaments' bases, or the effects of the contractile ring, my work demonstrates that a coarse description of individual cells is sufficient to explain the behavior that we observe in colonies. Compare this with \textit{Volvox}, which uses connections and communication between cells to control its inversion. One might imagine that the complexity of a molecular pathway for a single cell to exhibit phototaxis or regulate feeding/swimming efficiency could easily exceed the ring contraction as currently understood in \textit{C. flexa} \citep{brunet2019}. \mynote{cite Volvox inversion papers, cite papers on phototaxis or swimming signalling pathways}

\section{Discrete cell sheet topology}

The discrete model of \textit{C. flexa} demonstrates that deviations from a hexagonally packed sheet at as few as one cell are sufficient to induce substantial bending.
The geometric effects of such \textit{topological defects} are established in the context of crystal structures.

As discussed in the comparison between the continuous and discrete model, a topological defect at a cell can be likened to a different preferred Gaussian curvature in the continuous perspective. 
Indeed, \citet{seung1988} wrote a comparable model with defected lattices that deformed to an energy function given by stretching energies and bending energies.
There, the authors also found that 


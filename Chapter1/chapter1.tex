%!TEX root = ../thesis.tex
%*******************************************************************************
%*********************************** First Chapter *****************************
%*******************************************************************************

\chapter{Introduction} %Title of the First Chapter

\ifpdf
    \graphicspath{{Chapter1/Figs/Raster/}{Chapter1/Figs/PDF/}{Chapter1/Figs/}}
\else
    \graphicspath{{Chapter1/Figs/Vector/}{Chapter1/Figs/}}
\fi


%********************************** %First Section  **************************************
\section{Background} %Section - 1.1 

\begin{comment}

\nomenclature[z-cif]{$CIF$}{Cauchy's Integral Formula}                                % first letter Z is for Acronyms 
\nomenclature[a-F]{$F$}{complex function}                                                   % first letter A is for Roman symbols
\nomenclature[g-p]{$\pi$}{ $\simeq 3.14\ldots$}                                             % first letter G is for Greek Symbols
\nomenclature[g-i]{$\iota$}{unit imaginary number $\sqrt{-1}$}                      % first letter G is for Greek Symbols
\nomenclature[g-g]{$\gamma$}{a simply closed curve on a complex plane}  % first letter G is for Greek Symbols
\nomenclature[x-i]{$\oint_\gamma$}{integration around a curve $\gamma$} % first letter X is for Other Symbols
\nomenclature[r-j]{$j$}{superscript index}                                                       % first letter R is for superscripts
\nomenclature[s-0]{$0$}{subscript index}                                                        % first letter S is for subscripts
\end{comment}

%********************************** %Second Section  *************************************
\section{Choanoflagellates} %Section - 1.2

Primitive, leads to sponges

Choanoflagellates are increasingly studied as a model for understanding how multicellular animal life emerged. \citet{fairclough2010} shows that the transition from single cell to multicellular colony in \textit{S. rosetta} occurs by cell division, with cells remaining attached to each other. 

%********************************** % Third Section  *************************************
\section{\textit{Choanoeca flexa}}  %Section - 1.3 

Shape inversion: swimming/feeding, light triggered, dynamics of process
Connected by collars, (presumed) active transformation via contractile ring
Likely stiff actin collars with some intrinsic curvature and potentially different stiffness between the two states.

If other choanoflagellates are any indication, we know in \textit{S. rosetta} that cell division is asynchronous \citep{fairclough2010}. If \textit{C. flexa} forms colonies by a similar mechanism, then we might not expect regularity in its structure. \mynote{this need not be true! What would asynchronous cell division indicate about structure?}

\section{Thesis overview}

The thesis will be structured in this way.

%!TEX root = ../thesis.tex
%*******************************************************************************
%*********************************** First Chapter *****************************
%*******************************************************************************

\chapter{Introduction} %Title of the First Chapter

\ifpdf
    \graphicspath{{Chapter1/Figs/Raster/}{Chapter1/Figs/PDF/}{Chapter1/Figs/}}
\else
    \graphicspath{{Chapter1/Figs/Vector/}{Chapter1/Figs/}}
\fi


%********************************** %First Section  **************************************
\section{Background} %Section - 1.1 

\textbf{Thesis statement: studying evolutionarily basal organisms lets us study the simple ways that nature exploits geometry in living systems without higher order functional complications.}

\begin{comment}

\nomenclature[z-cif]{$CIF$}{Cauchy's Integral Formula}                                % first letter Z is for Acronyms 
\nomenclature[a-F]{$F$}{complex function}                                                   % first letter A is for Roman symbols
\nomenclature[g-p]{$\pi$}{ $\simeq 3.14\ldots$}                                             % first letter G is for Greek Symbols
\nomenclature[g-i]{$\iota$}{unit imaginary number $\sqrt{-1}$}                      % first letter G is for Greek Symbols
\nomenclature[g-g]{$\gamma$}{a simply closed curve on a complex plane}  % first letter G is for Greek Symbols
\nomenclature[x-i]{$\oint_\gamma$}{integration around a curve $\gamma$} % first letter X is for Other Symbols
\nomenclature[r-j]{$j$}{superscript index}                                                       % first letter R is for superscripts
\nomenclature[s-0]{$0$}{subscript index}                                                        % first letter S is for subscripts
\end{comment}

Discuss evolutionary origins of multicellularity and some basic examples like Volvox. What drives organisms to become multicellular? 

The organisms in the \textit{Volvox} genus are well studied for being a primitive example of multicellularity and a shining beacon of functional geometric changes. These organisms attach their cells to one another using an extra-cellular matrix, as \textit{S. rosetta} does as well (discussed later). 

We are interested in multicellularity at an evolutionarilty basal level to understand the basic reasons that life evolved to form multicellular organisms. Sponges are as basal as multicellular animal life goes \mynote{Discuss why and reference a review about sponges evolutionary simplicity. \citet{carr2008} could be a good ref to give here regarding choanoflagellates}. 

While we are interested in sponges since they are members of the animal kingdom, choanoflagellates are often considered to be evolutionarily and morphologically comparable. 



%********************************** %Second Section  *************************************
\section{Choanoflagellates} %Section - 1.2

\mynote{Discuss \citet{carr2008} about molecular phylogeny of choanoflagellates}

\citet{mah2014} offers the first comprehensive comparison between sponge choanocyte and choanoflagellate morphology. Sponge collars are fairly cylindrical while choanoflagellate collars are more cone-like. Choanoflagellates have glycocalyx, but seemingly around the cell body \citep{leadbeater2008}. Notably the collars in choanoflagellates are always microvillar and always present, while in sponges they emerge as a consequence of cell differentiation \mynote{cite this!}. \mynote{this is an old connection, though. James-Clark (1868) first describes the similarities. Tuzet (1963) finds that tehy have a common ancestor but not that sponges evolved from choanoflagellates. see references in first paragraph of \citet{leadbeater1983}.} 

Choanoflagellates are increasingly studied as a model for understanding how multicellular animal life emerged. \citet{fairclough2010} shows that the transition from single cell to multicellular colony in \textit{Salpingoeca rosetta} occurs by cell division, with cells remaining attached to each other. 

\textit{S. rosetta} has an extracellular matrix \citep{larson2020}. \citet{larson2020} finds that the extracellular matrix constrains cells to grow and divide to a given colony shape. This paper also finds that \textit{S. rosetta} does not have distinct cell lineages or a developmental plan.

\citet{kirkegaard2016} finds that collared choanoflagellates drive the most flow through their collars by swimming fastest, which occurs in the unicellular state \citep{michelin2011}. This makes it unclear that forming rosette colonies is for the sake of improved feeding. The authors point to evidence that \textit{S. rosetta} is induced to form rosette colonies by bacterial cues to suggest that the reasons for the development of multicellularity may be more subtle than previously expected \citep{alegado2012}.

%********************************** % Third Section  *************************************
\section{\textit{Choanoeca flexa}}  %Section - 1.3 

\citet{brunet2019} describe a newly discovered choanoflagellate, \textit{Choanoeca flexa}, which lives and feeds in aquatic environments. Here, I describe the relevant properties and characteristics of these cells and their colonies for modeling its structure and behavior. All descriptions proceed from \citet{brunet2019} and private communications with the authors.

Another sheet forming-choanoflagellate was described in \citet{leadbeater1983}, \textit{Proterospongia choanojuncta}. The literature is full of other choanoflagellates also forming colonies with curved geometries, i.e. \citet{lauterborn1898}. 

Nor is \textit{C. flexa} the first choanoflagellate observed to have a contractile process that influences collar angle. The only other member of \textit{Choanoeca}, \textit{C. perplexa}, has also been observed to feature a rapid, dramatic adduction/abduction of the collar microvilli between $10^\circ$ and $90^\circ$ from the apicobasal axis \citep{ellis1930}. \citet{leadbeater1977} described contractions at the base of the collars that occur with changes in the collar angle.
Remaining wary of the risk of speculating for \textit{C. flexa}, \citet{leadbeater1977} speculated that colonies of \textit{C. perplexa} form by daughter cells remaining attached after division. This is supported by the observed mechanism of individual cell division, where daughter cells are temporarily connected at a late stage by some collar microvilli. 

Shape inversion: swimming/feeding, light triggered, dynamics of process
Connected by collars, (presumed) active transformation via contractile ring. This differs from choanoflagellates like \textit{S. rosetta} which use ECM \citep{larson2020}. 
Likely stiff actin collars with some intrinsic curvature and potentially different stiffness between the two states.

If other choanoflagellates are any indication, we know in \textit{S. rosetta} that cell division is asynchronous \citep{fairclough2010}. If \textit{C. flexa} forms colonies by a similar mechanism, then we might not expect regularity in its structure. \mynote{this need not be true! What would asynchronous cell division indicate about structure?}

We are interested in this species because we hope we can use its colonies' geometries as a model for sponge choanocyte chambers and more broadly to understand how life most simply exploits shape at a multicellular level. It is quite difficult to experimentally model flows in the choanocyte chamber \citep{todo}, and published results on choanocyte flow have only been computational. Understanding the geometry of \textit{C. flexa} can contribute to understanding the flows that a colony can drive, which will contribute to the growing body of knowledge on flows driven by several pumps positioned along a surface \citep{asadzadeh2019}.

\section{Thesis overview}

The thesis will be structured in this way.
